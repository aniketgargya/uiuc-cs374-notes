\documentclass[12pt]{article}
\usepackage{../lecture}
\lecture{11}{Divide and Conquer}
\date{March 4, 2021}

\begin{document}
\maketitle

\section{Pivot}

\subsection{Quicksort}
\begin{itemize}
    \item[] \lstinputlisting{code/Quicksort.sudo}
    \item For a random pivot, we take the largest r, so the runtime is \\ 
    $T(n) = O(n) + T(r - 1) + T(n - r) \leq O(n) + T(0) + T(n - 1) = O(n^2)$
    \item If the pivot breaks the array into subarrays of length less than $\frac{3n}{4}$ and $\frac{n}{4}$, then the runtime is \\
    $T(n) = O(n) + T(\frac{3n}{4}) + T(\frac{n}{4}) = O(n\log n)$
    \item To obtain a good partition value, we will reduce the problem to the Selection problem. We will select the $\frac{n}{2}$ smallest element.
\end{itemize}

\subsection{Quickselect}
\begin{itemize}
    \item[] \lstinputlisting{code/Quickselect.sudo}
    \item As in Quicksort, a random pivot would have a rutime for $O(n^2)$
\end{itemize}

\subsection{BFPRT/Median of Medians}
\begin{itemize}
    \item Split $A[1...n]$ into $\frac{n}{5}$ chunks of size $5$.
    \item Find the median of each chunk in $O(1)$ time: $O(n)$ in total.
    \item Find the median of the medians, recursively: $T(\frac{n}{5})$ on the left if the previous MoM was to the right of the median, or the right if the previous MoM was to the left of the median.
    \item Use the median of the medians (MoM) as the pivot.
\end{itemize}

\subsection{Pivots}
\begin{itemize}
    \item In practice, $p = 1$ "good on average"
    \item In practice, random $p$ "very good" with high probability
    \item In theory, $p = \text{MoM}$ "excellent", slow in practice
\end{itemize}

\section{Karatsuba}
\begin{itemize}
    \item The runtime of multiplying two $n$ digit numbers is $O(n^2)$.
    \item Kolmogorov's Conjecture: $\Omega(n^2)$, Karatsuba: $O(n^2)$
    \item For $x \times y$, let $x = a \times 10^{\frac{n}{2}} + b, y = c \times 10^{\frac{n}{2}} + d$. \\
    $xy = (a \times 10^{\frac{n}{2}} + b)(c \times 10^{\frac{n}{2}} + d) = ac \times 10^n + (ad + bc) \times 10^{\frac{n}{2}} + bd$
    \item $T(n) = 4T(\frac{n}{2}) + O(n) \rightarrow O(n^2)$
    \item However $(a - b)(c - d) = ac - (ad + bc) + bd$
    \item So $T(n) = 3T(\frac{n}{2}) + O(n) \rightarrow \Theta(n^{\log_2 3})$
    \item[] \lstinputlisting{code/Fastmultiplication.sudo}
\end{itemize}

\section{Theory vs Practice}
\begin{itemize}
    \item $x, y \geq 1000$ -- Karatsuba
    \item $n \geq 10$ -- Splitting to 3
    \item $n \geq 1000$ -- FFT
    \item $n \geq 10^{100}$ -- $n\log n$ algorithm
\end{itemize}

\end{document} 
