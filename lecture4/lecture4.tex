\documentclass[12pt]{article}
\usepackage{../lecture}
\lecture{4}{NFAs}
\date{February 4, 2021}

\begin{document}
\maketitle

\section{Nondeterministic-Finite-Autmata (NFA) Introduction}

\subsection{NFA by Example}
\begin{itemize}
    \item When you come to a fork in the road, take it.
    \item NFAs are automata whose logic is \textit{not} deterministic.
\end{itemize}

\subsection{Non-determinism in Computing}
\begin{itemize}
    \item \textbf{Non-determinism} is a special property of algorithms.
    \item An algorithm that is capable of taking multiple states concurrently. Whenever it reaches a choice, it takes both paths.
    \item If there is a path for the string to be accepted by the machine, then the string is part of the language.
\end{itemize}

\subsection{NFA Acceptance: Informal}
\begin{itemize}
    \item \textbf{Informal Definition}: An NFA $N$ accepts a string $w$ iff some accepting state is reached by $N$ from the start state on input $w$.
    \item The language accepted (or recognized) by a NFA $N$ is denoted by $L(N)$ and defined as: $L(N) = \{ w \mid N \text{ accepts } w \}$
    \item Unlike DFAs, it is easier in NFAs to show that a string is accepted than to show that a string is \textit{not} accepted.
\end{itemize}

\section{Formal Definition of NFA}

\subsection{Formal Tuple Notation}
\begin{itemize}
    \item A \textbf{non-deterministic finite automata (NFA)} $N = (Q, \sum, \delta, s, A)$ is a five tuple where
    \begin{itemize}
        \item $Q$ is a finite set whose elements are called \textbf{states}
        \item $\sum$ is called the \textbf{input alphabet}
        \item $\delta: Q \times \sum \cup \{ \epsilon \} \rightarrow \mathcal{P}(Q)$ is the \textbf{transition function} (here $\mathcal{P}(Q)$ is the power set of $Q$)
        \item $s \in Q$ is the \textbf{start state}
        \item $A \subseteq Q$ is the set of \textbf{accepting/final} states
    \end{itemize}
    \item $\delta(q, a)$ for $a \in \sum \cup \{ \epsilon \}$ is a subset of $Q$ -- a set of states.
\end{itemize}

\subsection{Extending the Transition Function to Strings}
\begin{itemize}
    \item NFA $N = (Q, \sum, \delta, s, A)$
    \item $\delta(q, a)$: \textbf{set of states} that $N$ can go to from $q$ on reading $a \in \sum \cup \{ \epsilon \}$
    \item $\delta^{\ast}: Q \times \sum^{\ast} \rightarrow \mathcal{P}(Q)$: set of states reachable on input $w$ starting in state $q$.
    \item For $X \subseteq Q: \epsilon reach(X) = \bigcup_{x \in X}\epsilon reach(x)$.
    \item $\epsilon reach(q)$: set of all states that $q$ can reach using only $\epsilon \text{-transitions}$.
    \item Inductive definition of $\delta^{\ast} : Q \times \sum^{\ast} \rightarrow \mathcal{P}(Q)$:
    \begin{itemize}
        \item if $w = \epsilon$, $\delta^{\ast}(q, w) = \epsilon reach(q)$
        \item if $w = a$ where $a \in \sum$: $\delta^{\ast}(q, a) = \epsilon reach(\underset{p \in \epsilon reach(q)}{\bigcup} \delta(p, a))$
        \item if $w = ax$: $\delta^{\ast}(q, w) = \epsilon reach ( \underset{p \in \epsilon reach(q)}{\bigcup}(\underset{r \in \delta^{\ast} (p, a)}{\bigcup} \delta^{\ast}(r, x)))$
    \end{itemize}
\end{itemize}

\subsection{Formal Acceptance of Language by $N$}
\begin{itemize}
    \item A string $w$ is accepted by NFA $N$ if $\delta_N^{\ast}(s, w) \cap A \neq \emptyset$.
    \item The language $L(N)$ accepted by a NFA $N = (Q, \sum, \delta, s, A)$ is
    \begin{equation}
        \textstyle \{ w \in \sum^{\ast} \mid \delta^{\ast}(s, w) \cap A \neq \emptyset \}
    \end{equation}
\end{itemize}

\section{Why non-determinism?}
\begin{itemize}
    \item Non-determinism adds power to the mode; richer programming language and hence (much) easier to "design" programs.
    \item Fundamental in \textbf{theory} to prove many theorems.
    \item Very important in \textbf{practice} directly and indirectly.
    \item Many deep connections to various fields in Computer Science and Mathematics.
\end{itemize}

\section{Constructing NFAs}

\subsection{DFAs and NFAs}
\begin{itemize}
    \item Every DFA is an NFA so NFAs are at least as powerful as DFAs.
    \item NFAs prove the ability to "guess and verify" which simplifies design and reduces number of stats.
    \item Easy proofs of some closure properties.
\end{itemize}

\subsection{A Simple Transformation}
\begin{itemize}
    \item Theorem: For every NFA $N$, there is another NFA $N'$ such that $L(N) = L(N')$ and such that $N'$ has the following two properties:
    \begin{itemize}
        \item $N'$ has a single final state $f$ that has no outgoing transitions.
        \item The start state $s$ of $N$ is different from $f$.
    \end{itemize}
\end{itemize}

\subsection{Closure Properties of NFAs}
\begin{itemize}
    \item Theorem: For any two NFAs $N_1$ and $N_2$ there is an NFA $N$ such that $L(N) = L(N_1) \cup L(N_2)$.
    \item Theorem: For any two NFAs $N_1$ and $N_2$ there is an NFA $N$ such that $L(N) = L(N_1) \cdot L(N_2)$.
    \item Theorem: For any NFA $N_1$ there is an NFA $N$ such that $L(N) = (L(N_1))^{\ast}$.
\end{itemize}

\end{document}
