\documentclass[12pt]{article}
\usepackage{../lecture}
\lecture{2}{Regular Languages}
\date{January 28, 2021}

\begin{document}
\maketitle

\section{Regular Languages}

\subsection{Theorem (Kleene's Theorem)}
\begin{itemize}
    \item A language is regular if and only if it can be obtained from finite languages by applying the following three operations a finite number of times:
    \begin{enumerate}
        \item Union
        \item Concatenation
        \item Repetition 
    \end{enumerate}
\end{itemize}

\subsection{Regular Languages}
\begin{itemize}
    \item A class of simple but useful languages.
    \item The set of \textbf{regular languages} over some alphabet $\sum$ is defined inductively:
    \begin{itemize}
        \item Base Case:
        \begin{itemize}
            \item $\emptyset$ is a regular language.
            \item $\{\epsilon\}$ is a regular language.
            \item $\{ \alpha \}$ is a regular language for each $a \in \sum$. Interpreting $a$ as string of length 1.
        \end{itemize}
        \item Inductive Step:
        \begin{itemize}
            \item We can build up languages using a few basic operations:
            \begin{itemize}
                \item If $L_1$, $L_2$ are regular, then $L_1 \cup L_2$ is regular.
                \item If $L_1$, $L_2$ are regular, then $L_1 L_2$ is regular.
                \item If $L$ is regular, then $L^{\ast} = \bigcup_{n \geq 0} L^n$ is regular. The $\ast$ operator is the \textit{Kleene star}.
                \item If $L$ is regular, then so is $\bar{L} = \sum^{\ast} \setminus L$.
            \end{itemize}
            \item Regular languages are \textbf{closed} under \textbf{operations} of union, concatenation and Kleene star.
        \end{itemize}
    \end{itemize}
    \item \textbf{Important}: Any language generated by a finite sequence of such operations is regular.
    \item Lemma: Let $L_1, L_2, ...$ be regular languages over alphabet $\sum$. Then the language $\bigcup_{i=1}^{\infty} L_i$ is not necessarily regular.
\end{itemize}

\subsection{Some Simple Regular Languages}
\begin{itemize}
    \item If $w$ is a string, then $L = \{ w \}$ is regular.
    \item Lemma: Every finite language $L$ is regular.
\end{itemize}

\section{Regular Expressions}

\subsection{Regular Expressions}
\begin{itemize}
    \item A way to denote regular languages
    \item Simple \textbf{patterns} to describe related strings
    \item Useful in:
    \begin{itemize}
        \item text searchers (editors, Unix/grep, emacs)
        \item compilers: lexical analysis
        \item compact way to represent interesting/useful languages
        \item dates back to 50's: Stephen Kleene who has a star named after him.
    \end{itemize}
\end{itemize}

\subsection{Inductive Definition}
\begin{itemize}
    \item A \textbf{regular expression} $r$ over an alphabet $\sum$ is one of the following:
    \begin{itemize}
        \item \textbf{Bases cases}:
        \begin{itemize}
            \item $\emptyset$ denotes the language $\emptyset$.
            \item $\epsilon$ denotes the language $\{\epsilon\}$.
            \item $a$ denotes the language $\{a\}$.
        \end{itemize}
        \item \textbf{Inductive cases}: If $r_1$ and $r_2$ are regular expressions denoting languages $R_1$ and $R_2$ respectively, then:
        \begin{itemize}
            \item $(r_1 + r_2)$ denotes the language $R_1 \cup R_2$.
            \item $(r_1 \cdot r_2) = (r_1r_2)$ denotes the language $R_1R_2$.
            \item $(r_1)^{\ast}$ denotes the language $R_1^{\ast}$.
        \end{itemize}
    \end{itemize}
\end{itemize}

\subsection{Notation and Parenthesis}
\begin{itemize}
    \item For a regular expressions $r$, $L(r)$ is the language denoted by $r$. Multiple regular expressions can denote the same language.
    \item Two regular expressions, $r_1$ and $r_2$ are \textbf{equivalent} if $L(r_1) = L(r_2)$.
    \item Omit parenthesis by adopting precedence order: $\ast$, concatenate, $+$. $r^{\ast} s + t = ((r^{\ast})s)+t$.
    \item Omit parenthesis by associativity of each of these operations.
    \item \textbf{Superscript $+$}. For convenience, define $r^+ = rr^{\ast}$. Hence, if $L(r) = R$, then $L(r^+) = R^+$.
\end{itemize}

\end{document}