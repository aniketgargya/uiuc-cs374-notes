\documentclass[12pt]{article}
\usepackage{../lecture}
\lecture{3}{DFAs}
\date{February 2, 2021}

\begin{document}
\maketitle

\section{Deterministic-Finite-Autmata (DFA)}

\subsection{DFAs also called Finite State Machines (FSMs)}
\begin{itemize}
    \item State machines that are common in practice:
    \begin{itemize}
        \item Vending machines
        \item Elevators
        \item Digital watches
        \item Simple network protocols
    \end{itemize}
    \item Programs with fixed memory.
\end{itemize}

\section{Graphical Representation of DFA}

\subsection{Graphical Representation/State Machine}
\begin{itemize}
    \item Directed graph with nodes represent \textbf{states} and edge/arcs representing \textbf{transitions} labeled by symbols in $\sum$.
    \item For each state (vertex) $q$ and symbol $a \in \sum$ there is \textit{exactly} one outgoing edge labeled by $a$.
    \item Initial/start state has a pointer (or labeled as $s$, $q_0$, or "start").
    \item Some states with double circles labeled as accepting/final states.
    \item A DFA $M$ accepts a string $w$ iff the unique walk starting at the start state and spelling out $w$ ends in an accepting state.
    \item The language accepted (or recognized) by a DFA $M$ is denoted by $L(M)$ and defined as $L(M) = \{ w \mid M$ accepts $w \}$.
\end{itemize}

\section{Formal Definition of DFA}

\subsection{Formal Tuple Notation}
\begin{itemize}
    \item A \textbf{deterministic finite automata (DFA)} $M = (Q, \sum, \delta, s, A)$ is a five tuple where:
    \begin{itemize}
        \item $Q$ is a finite set whose elements are called \textbf{states}
        \item $\sum$ is a finite set called the \textbf{input alphabet}
        \item $\delta$: $Q \times \sum \rightarrow Q$ is the \textbf{transition function}
        \item $s \in Q$ is the \textbf{start state}
        \item $A \subseteq Q$ is the set of \textbf{accepting/final} states
    \end{itemize}
    \item Common alternate notation: $q_0$ for start state, $F$ for final states.
\end{itemize}

\section{Extending the Transition Function to Strings}
\begin{itemize}
    \item Given DFA $M = (Q, \sum, \delta, s, A)$, $\delta(q, a)$ is the state that $M$ goes to from $q$ on reading letter $a$.
    \item Useful to have notation to specify the unique state that $M$ will reach from $q$ on reading string $w$.
    \item Transition function $\delta^{\ast}$: $Q \times \sum^{\ast} \rightarrow Q$ defined inductively as follows:
    \begin{itemize}
        \item $\delta^{\ast}(q,w) = q$ if $w = \epsilon$
        \item $\delta^{\ast}(q,w) = \delta^{\ast}(\delta(q,a),x)$ if $w = ax$
    \end{itemize}
\end{itemize}

\section{Formal Definition of Language Accepted by $M$}
\begin{itemize}
    \item The language $L(M)$ accepted by a DFA $M = (Q, \sum, \delta, s, A)$ is
    \begin{equation}
        \textstyle \{ w \in \sum^{\ast} \mid \delta^{\ast}(s, w) \in A \}
    \end{equation}
\end{itemize}

\section{Constructing DFAs}
\begin{itemize}
    \item How do we design a DFA $M$ for a given language $L$? That is L(M) = L.
    \item DFA is like a program that has a fixed amount of memory independent of input size.
    \item The memory of a DFA is encoded in its states.
    \item The state/memory must capture enough information from the input seen so far that it is sufficient for the suffix that is yet to be seen (not that DFA cannot go back).
\end{itemize}

\section{Constructing Regular Expressions}

\subsection{State Removal Method}
\begin{itemize}
    \item If $q_1 = \delta(q_0, x)$ and $q_2 = \delta(q_1, y)$, then $q_2 = \delta(q_1, y) = \delta(\delta(q_0, x), y) = \delta(q_0, xy)$.
    \item Keep removing states, and replace them with other, more complicated regular expressions states.
\end{itemize}

\subsection{Algebraic Method}
\begin{itemize}
    \item Transition functions themselves are algebraic expressions.
    \item Demarcrate states as variables.
    \item Can rewrite $q_1 = \delta(q_0, x)$ as $q_1 = q_0x$.
    \item Solve for accepting state.
    \item Theorem (Arden's Theorem): $R = Q + RP = QP^{\ast}$.
\end{itemize}

\section{Product Construction}

\subsection{Union and Intersection}
\begin{itemize}
    \item Are languages accepted by DFAs closed under union? That is given DFAs $M_1$ and $M_2$ is there a DFA that accepts $L(M_1) \cup L(M_2)$?
    \item How about intersection $L(M_1) \cap L(M_2)$?
    \item Idea from programming: on input string $w$
    \begin{itemize}
        \item Simulate $M_1$ on $w$
        \item Simulate $M_2$ on $w$
        \item If both accept, then $w \in L(M_1) \cap L(M_2)$. If at least one accepts, then $w \in L(M_1) \cup L(M_2)$.
        \item \textbf{Catch:} We want a single DFA $M$ that can only read $w$ once. This is done by multiplying DFAs.
    \end{itemize}
\end{itemize}

\subsection{Product Construction for Intersection}
\begin{itemize}
    \item $M_1 = (Q_1, \sum, \delta_1, s_1, A_1)$ and $M_2 = (Q_2, \sum, \delta_2, s_2, A_2)$
    \item \textbf{Theorem}: $L(M) = L(M_1) \cap L(M_2)$.
    \item Create $M = (Q, \sum, \delta, s, A)$ where
    \begin{itemize}
        \item $Q = Q_1 \times Q_2 = \{ (q_1, q_2) \mid q_1 \in Q_1, q_2 \in Q_2 \}$
        \item $s = (s_1, s_2)$
        \item $\delta$: $Q \times \sum \rightarrow Q$ where $\delta((q_1, q_2), a) = (\delta_1(q_1, a), \delta_2(q_2, a))$
        \item $A = A_1 \times A_2 = \{ (q_1, q_2) \mid q_1 \in A_1, q_2 \in A_2 \}$
    \end{itemize}
\end{itemize}

\subsection{Product Construction for Union}
\begin{itemize}
    \item $M_1 = (Q_1, \sum, \delta_1, s_1, A_1)$ and $M_2 = (Q_2, \sum, \delta_2, s_2, A_2)$
    \item \textbf{Theorem}: $L(M) = L(M_1) \cup L(M_2)$.
    \item Create $M = (Q, \sum, \delta, s, A)$ where
    \begin{itemize}
        \item $Q = Q_1 \times Q_2 = \{ (q_1, q_2) \mid q_1 \in Q_1, q_2 \in Q_2 \}$
        \item $s = (s_1, s_2)$
        \item $\delta$: $Q \times \sum \rightarrow Q$ where $\delta((q_1, q_2), a) = (\delta_1(q_1, a), \delta_2(q_2, a))$
        \item $A = A_1 \times A_2 = \{ (q_1, q_2) \mid q_1 \in A_1 $ or $ q_2 \in A_2 \}$
    \end{itemize}
\end{itemize}

\end{document}
