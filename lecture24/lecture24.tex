\documentclass[12pt]{article}
\usepackage{../lecture}
\lecture{24}{Hamiltonian Cycle, 3-Color, Circuit-SAT}
\date{April 27, 2021}

\begin{document}
\maketitle

\section{NP-Completeness of Hamiltonian Cycle}

\subsection{Directed Hamiltonian Cycle}
\begin{itemize}
    \item Input: A directed graph $G = (V, E)$ with $n$ vertices.
    \item Output: Does $G$ have a Hamiltonian cycle?
    \begin{itemize}
        \item A Hamiltonian cycle is a cycle in the graph that visits every vertex in $G$ exactly once.
    \end{itemize}
\end{itemize}

\subsection{Reduction}
\begin{itemize}
    \item Given a 3SAT formula $\varphi$, we need to create a graph $G_\varphi$ such that
    \begin{itemize}
        \item $G_\varphi$ has a Hamiltonian cycle if and only if $\varphi$ is satisfiable.
        \item $G_\varphi$ should be constructible from $\varphi$ by a polynomial time algorithm $\mathcal{A}$.
    \end{itemize}
    \item Notation: $\varphi$ has $n$ variables $x_1, x_2, ..., x_n$ and $m$ clauses $C_1, C_2, ..., C_m$.
    \item We can traverse path $i$ from left to right if and only if $x_i$ is set to true.
    \item Each path has $3(m + 1)$ nodes where $m$ is the number of clauses in $\varphi$; nodes are numbered from left to right ($1$ to $3m + 3$).
\end{itemize}

\end{document}
