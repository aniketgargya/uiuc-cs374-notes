\chapterWithSubtitle{Backtracking}{March 9, 2021}

\section{Recursion Types}

\begin{itemize}
    \item \textbf{Divide and Conquer}:
    \begin{itemize}
        \item Problem reduced to multiple \textit{independent} sub-problems.
        \item Each sub-problem is a fraction smaller.
    \end{itemize}
    \item \textbf{Backtracking}:
    \begin{itemize}
        \item A sequence of decision problems. Recursion tries all possibilities at each step.
        \item Each sub-problem is only a constant smaller (such as from $n$ to $n - 1$)
    \end{itemize}
\end{itemize}

\section{N Queens Problem}

\subsection{Objective}
\begin{itemize}
    \item Place $n$ queens on an $n \times n$ board so that no two queens are attacking each other.
\end{itemize}

\subsection{Redefining as a Recursive Problem}
\begin{itemize}
    \item For a row-by-row method, the base cases are:
    \begin{itemize}
        \item when any position in the row is attacked by a queen on an earlier row, recursion terminates.
        \item when all $n$ queens are placed.
    \end{itemize}
    \item When backtracking, one must:
    \begin{itemize}
        \item change the problem into a sequence of decision problems.
        \item try each possibility for the current decision.
        \item let the recursion fairy make all remaining decisions.
    \end{itemize}
    \item To redefine the problem to make the recursion work:
    \begin{itemize}
        \item the recursion does not solve the $n - 1$ queens problem.
        \item one needs to place the $r$-th queen so that it is not attacked by a queen on an earlier row.
    \end{itemize}
    \item The recursive subproblem:
    \begin{itemize}
        \item Input: $r - 1$ queens placed in earlier rows.
        \item Place the remaining $n - r + 1$ queens, one on each row.
        \item Recurse by increasing $r$.
    \end{itemize}
\end{itemize}
