\chapterWithSubtitle{Bellman-Ford, Dynamic Programming on DAGs}{April 1, 2021}

\section{Negative Edges: Bellman-Ford}

\subsection{Dijkstra's Algorithm and Negative Lengths}
\begin{itemize}
    \item With negative length edges, Dijkstra's algorithm can fail.
    \item It is based on the assumption that if $s = v_0 \rightarrow v_1 \rightarrow v_2  \rightarrow ... \rightarrow v_k$ is a shortest path from $s$ to $v_k$, then $dist(s, v_i) \leq dist(s, v_{i + 1})$ for $0 \leq i < k$. The assumption only holds true for non-negative edge lengths.
\end{itemize}

\subsection{What Can We Learn from Dijkstra?}
\begin{itemize}
    \item However, $d'(s, u) = \min(d'(s, u), dist(s, v) + l(v, u))$ because $d'(s, u) \geq d(s, u)$, still holds true.
    \item If $s = v_0 \rightarrow v_1 \rightarrow v_2 \rightarrow ... \rightarrow v_k$ is a shortest path from $s to v_k$, then
    \begin{itemize}
        \item for $1 \leq i < k$, $s = v_0 \rightarrow v_1 \rightarrow v_2 \rightarrow ... \rightarrow v_i$ is a shortest path from $s$ to $v_i$. In other words, a subpath of a shortest path is still a shortest path.
        \item it is not necessarily true that $dist(s, v_i) \leq dist(s, v_{i + 1})$. The intermediate set is no longer $X$; in fact, it can be anything.
    \end{itemize}
    \item Solution: Update all edges $\left|V\right| - 1$ times.
\end{itemize}

\subsection{Bellman-Ford Algorithm}
\begin{itemize}
    \item[] \lstinputlisting{lecture19/code/bellman-ford.sudo}
    \item The running time for this algorithm is $O(mn)$.
\end{itemize}

\section{Bellman-Ford and Dynamic Programming}

\subsection{Shortest Paths and Recursion}
\begin{itemize}
    \item We can compute the shortest path distance from $s$ to $t$ recursively.
    \item Let $G$ be a directed graph with arbitrary edge lengths. If $s = v_0 \rightarrow v_1 \rightarrow ... \rightarrow v_k$ is a shortest path from $s$ to $v_k$, then for $1 \leq i < k$, $s = v_0 \rightarrow v_1 \rightarrow ... \rightarrow v_i$ is a shortest path from $s$ to $v_i$.
    \item Our sub-problem can be path of fewer hops/edges.
\end{itemize}

\subsection{Hop Based Recursion: Bellman-Ford Algorithm}
\begin{itemize}
    \item $d(v, k)$ is the shortest path length from $s$ to $v$ using at most $k$ edges. Note: $dist(s, v) = d(v, n - 1)$.
    \item Recursion for $d(v, k)$: \begin{equation}
        d(v, k) = \min \left\{
            \begin{tabular}{c}
                $\min_{u \in In(v)}(d(u, k - 1) + l(u, v))$ \\
                $d(v, k - 1)$ \\
            \end{tabular}
        \right\}
    \end{equation}
    Base case: $d(s, 0) = 0$ and $d(v, 0) = \infty$ for all $v \neq s$.
    \item This gives us the following algorithm
    \item[] \lstinputlisting{lecture19/code/bellman-ford-dp.sudo}
    \item This has a running time of $O(mn)$ and requires $O(n^2)$ space.
    \item We only depend on the previous column so we can reduce the space to $O(n)$.
    \item We are also unnecessarily copying the previous value to the current column, so we can just get rid of the memoization grid and we get the orignal Bellman-Ford Algorithm.
    \item[] \lstinputlisting{lecture19/code/bellman-ford.sudo}
\end{itemize}

\subsection{Negative Length Cycles}
\begin{itemize}
    \item \textbf{Negative Length Cycle}: A cycle $C$ where the sum of the edge lengths of $C$ is negative.
    \item Given $G = (V, E)$ with edge lengths and $s, t$. Suppose $G$ has a negative length cycle $C$, and $s$ can reach $C$ and $C$ can reach $t$. The shortest distance from $s$ to $t$ is $-\infty$.
    \item We can modify the Bellman-Ford algorithm to detect negative cycles. We run the loop once more, and if any of the distance reduce, then there must be a negative cycle. That would mean a path with $n$ hops is shorter than a path with $n - 1$ hops.
    \lstinputlisting{lecture19/code/bellman-ford-negative-cycle-detection.sudo}
    \item On a directed graph $G$, Bellman-Ford checks whether there is a negative cycle $C$ that is reachable from a specific vertex $s$. There may be negative cycles not reachable from $s$.
    \item Instead of running Bellman-Ford $\left|V\right|$ times (once on each node), we just add a new node $s'$ and connect it to all nodes of $G$ with zero length edges. We can then just run Bellman-Ford once on $s'$.
\end{itemize}
