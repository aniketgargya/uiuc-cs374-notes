\chapterWithSubtitle{Regular Languages - Closure Properties}{February 11, 2021}

\section{Closure Properties}
\begin{itemize}
    \item (Informal Definition) A set $A$ is \textbf{closed} under an operation \textbf{op} to any elements of $A$ results in an element that also belongs to $A$.
    \item Examples:
    \begin{itemize}
        \item Integers: closed under $+$, $-$, $\times$, but not under $/$.
        \item Postive Integers: closed $+$, but not under $-$.
        \item Regular languages: closed under union, intersection, Kleene star, complement, difference, homomorphism, inverse homomorphism, reverse, ...
    \end{itemize}
    \item Three broad approaches to prove that regular languages are closed under some new operation:
    \begin{itemize}
        \item Use existing closure properties.
        \item Transform regular expressions.
        \item Transform DFAs to NFAs -- versatile technique and shows the power of nondeterminism.
    \end{itemize}
\end{itemize}

\subsection{Homomorphism Closure}
\begin{itemize}
    \item A \textbf{homomorphism} is a cipher that is a one-to-one mapping to one character set to another.
    \item How do we prove $h(L)$ is regular if $L$ is regular? \\
    $
    h(x) = \begin{cases}
        1 & x = 0 \\
        0 & x = 1
    \end{cases}
    $
    \begin{enumerate}
        \item Suppose $R$ is a regular expression for $L$.
        \item We define $\text{Flip}(L) = L^F$ as a regular expression based off the regular expression for $L$ (using a finite number of concatenations, unions, and Kleene Star).
        \item Thus $L^F$ is regular because it has a regular expression.
    \end{enumerate}
    \item Thus we reduce the argument to $L(h(R)) = h(L(R))$.
    \item Let's define the regular expression $R^F$ inductively by transforming the operations in $R$. We see that:
    \begin{itemize}
        \item \textbf{Base Case}: Zero operators in $R$ means that $R =: a \in \sum, \epsilon, \emptyset$. In any case we define $R^F = h(R)$.
        \item Otherwise $R$ has three potential types of operators to transform. Splitting $R$ at an operator we see:
        \begin{itemize}
            \item $h(R_1 R_2) = h(R_1) \cdot h(R_2)$
            \item $h(R_1 \cup R_2) = h(R_1) \cup h(R_2)$
            \item $h(R_1^{\ast}) = (h(R))^{\ast}$
        \end{itemize}
    \end{itemize}
    \item Hence, since we can define $R^F$ via a regular language, $L^F$ is regular.
\end{itemize}

\subsection{Reverse Closure}
\begin{itemize}
    \item Given string $w$, $w^R$ is the reverse of $w$.
    \item For a language $L$, define $L^R = \{ w^R \mid w \in L \}$ as reverse of $L$.
    \item \textit{Theorem}: $L^R$ is regular if $L$ is regular.
    \begin{enumerate}
        \item Take some finite representation of $L$ such as regular expression $r$.
        \item Describe an algorithm $A$ that takes $r$ as input and outputs a regular expression $r'$ such that $L(r') = (L(r))^R$.
        \item Come up with $A$ and prove its correctness.
    \end{enumerate}
    \item How do we create a regular expression $r'$ for $L^R$?
    \begin{itemize}
        \item Suppose $r$ is a regular expressions for $L$. We create a regular expression $r'$ for $L^R$ inductively based on a recursive definition of $r$.
        \begin{itemize}
            \item \textbf{Base Case}: $r = \emptyset$ or $r = a$ for some $a \in \sum$, then $r' = r$
            \item $r = r_1 + r_2 \implies r' = r_1 + r_2$
            \item $r = r_1 \cdot r_2 \implies r' = r_2' \cdot r_1'$
            \item $r = (r_1)^{\ast} \implies r' = (r_1')^{\ast}$
        \end{itemize}
    \end{itemize}
    \item How do we create a machine $N$ for $L^R$?
    \begin{itemize}
        \item Given DFA $M = (Q, \sum, \delta, s, A)$, we want NFA $N$ such that $L(N) = (L(M))^R$.
        \item $N$ should accept $w^R$ iff $M$ accepts $w$.
        \item $M$ accepts $w$ iff $\delta_M^{\ast}(s,w) \in A$.
    \end{itemize}
\end{itemize}
